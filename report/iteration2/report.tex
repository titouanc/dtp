\section{Rapport de fin d'itération}

La physionomie de l'itération 2 a été intrinsèquement différente de celle de son 
antérieure. Nous n'avons eu la confirmation que nous pouvions développer
l'histoire 2 que 8 jours avant la fin de l'itération 2, ce qui a entraîné un 
rush plutôt conséquent. Toutefois, Nous y avions bien entendu déjà réfléchi et 
profité du début de l'itération pour entreprendre quelques refactorings 
rafraîchissants.\\

Anecdotiquement, on peut noter que c'est la troisième fois qu'un des membres de
l'équipe est malade et ne peut participer activement au développement pendant
plusieurs jours. La communication du groupe est restée forte et ces accrocs 
n'ont jamais eu d'impact; nous avons toujours réussi à rééquilibrer les 
responsabilités sur les épaules de tous les membres.

\subsection{Suivi du planning}


\subsection{Architecture}
	
	\subsubsection{Changements dans le modèle}
	% TODO FILL

	\subsubsection{Changements dans la GUI}

		\paragraph{CameraContext}
		Le pattern state permet de gérer l'état des contrôleurs 2D et 3D de la caméra. Nous n'avons donc pas notre propre classe caméra, nous utilisons directement celle créée de la classe SimpleApplication. Afin de géré l'état, il nous a donc été nécessaire de créer une classe supplémentaire CameraContext. Cette classe sert de listener de tout les événement liés au interaction avec la caméra.

		\paragraph{Lancement des contrôleurs}
		Nous avons légèrement amélioré la manière dont nous lançons un contrôleur.
		Ils possèdent désormais une méthode \texttt{run()} qu'il faut appeler
		après leur construction; cela nous permet de les lancer sans instancier
		de vue (pratique pour les tests, améliore le découplage). De plus, si 
		quelque chose se passe mal dans cette fonction, on reste sûr que 
		l'objet contrôleur a été bien construit, puisque son constructeur s'est
		terminé bien avant. 

	\subsubsection{Rajouts dans la GUI}

		\paragraph{Texture Panel}
		Il a fallu rajouter un panneau pour afficher les textures disponibles et
		offrir une interface à l'utilisateur pour en rajouter. Grâce à 
		l'architecture MVC que nous avions mise en place lors de la précédente 
		itération, c'était très facile et il a suffi de rajouter une TextureView
		et un TextureController.

		\paragraph{Création d'objets}
		Pour l'instant nous avons ajouté tout ce qui permet de manipuler les objets dans les contrôleurs existants. Les vues tel que la treeView et le canvas jmonkey sont réutilisé pour l'édition/affichage des objets. Par manque de temps nous n'avons pas pu fournir de solution architecturalement plus propre, cependant nous sommes conscient que le design pattern state pourrait largement améliorer et simplifier notre architecture. Nous avons comme projet de l'implémenter au plus vite avant de commencer la prochaine itération.

	\subsubsection{Architecture complète}
	Voici les diagrammes complets pour cette itération.


\subsection{Bonnes pratiques - Outils}

Nous avons continué à utiliser toutes les bonnes pratiques listées dans le
rapport de l'itération précédente. Nous avions toutefois omis de préciser que
nous utilisons \textbf{Trello} pour nous aider à identifier, assigner, séparer 
et préciser les tâches qui nous attendent. 

\subsection{Conclusion - What's next}

