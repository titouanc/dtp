\subsubsection{Enregistrement d'un projet}

\textbf{Durée estimée: 20h} \textbf{Difficulté estimée: 2/3} \textbf{Assigné à: Titouan}

Les projets d'architecture doivent être enregistrables dans des fichiers, et il doit être possible de charger un projet depuis un fichier. Différents éléments doivent être enregistrés, comme la configuration du projet (nom, auteur, ...), ainsi que des éléments affichables (polyèdres pour le moment, éventuellement les tags par la suite).

Les projets doivent être enregistrés séparément et sélectionnables dans un navigateur de fichiers pour ouverture ou enregistrement.
\subsubsection{Interface graphique}

\textbf{Durée estimée: 30h} \textbf{Difficulté estimée: 1/3} \textbf{Assigné à: Julian et Pierre}

L'interface graphique sera composé d'un menu, d'une barre d'outils et d'un écran splitter en deux. Sur cet écran splitter il y aura à gauche les objet utilisés et a droite le rendu 2D et 3D.
\subsubsection{Affichage du monde 2D et 3D}

\textbf{Durée estimée: 10h} \textbf{Difficulté estimée: 1/3} \textbf{Assigné à: Florentin, Bruno et Walter}

L'utilisateur devra pouvoir choisir d'afficher le plan d'architecture autant en 3D qu'en 2D. Chaque pièce pourra avoir une forme différente et n'aura pas forcément la forme d'un rectangle.
\subsubsection{Navigation dans le monde}

\textbf{Durée estimée: 5h} \textbf{Difficulté estimée: 1/3} \textbf{Assigné à: Bruno et Julian}

Dans la vue 3D, l'utilisateur devra avoir le contrôle de la caméra, que ce soit pour zoomer ou déplacer la caméra sur les cotés, et ce, aussi bien grâce à la souris que grâce aux flèches.
\subsubsection{Modification de la geometrie d'une piece}

\textbf{Durée estimée: 40h} \textbf{Difficulté estimée: 3/3} \textbf{Assigné à: Walter, Titouan, Bruno, Florentin}

L'utilisateur a accès à une vue d'édition lui permettant de rajouter ou éditer des murs et des étages. 

Tous les points du "sol" sont à la même hauteur, et à chacun d'eux est associée une élévation.

La hauteur de l'étage est déterminée par son point le plus haut. Le sol de l'étage suivant est égal à la hauteur du point le plus haut de l'étage précédent
\subsubsection{Creation d'un projet de demo}

\textbf{Durée estimée: 5h} \textbf{Difficulté estimée: 1/3} \textbf{Assigné à: Walter et Julian}

Lorsque l'utilisateur lance pour la première fois le programme EA3D, un projet de démonstration complet s'ouvre, pour lui montrer tout ce qu'il est possible de faire.
\subsubsection{Enregistrement automatique d'un projet}

\textbf{Durée estimée: 2h} \textbf{Difficulté estimée: 1/3} \textbf{Assigné à: Titouan et Pierre}

Le projet sur lequel travaille l'utilisateur est sauvé automatiquement sur le disque. L'utilisateur peut aussi enregistrer une copie du projet, auquel cas un nouveau fichier de projet est créé, et devient le projet courant.

Par défaut, lorsque l'utilisateur ouvre le programme, le dernier projet utilisé lui est présenté.
\subsubsection{Intégration de la vue 3D dans la GUI}

\textbf{Durée estimée: 10h} \textbf{Difficulté estimée: 2/3} \textbf{Assigné à: Pierre}

Il faut intégrer jmonkey qui s'occupe de la 3d dans swing qui s'occupe de l'interface graphique

http://hub.jmonkeyengine.org/wiki/doku.php/jme3:advanced:swing\_canvas
